\documentclass[11pt]{amsart}
\frenchspacing

%------------------------------------------------------------
% Packages
%------------------------------------------------------------
\usepackage{amsfonts}							%Special fonts
\usepackage{amsmath}							%align* environment
\usepackage{amssymb}							%Symbols
\usepackage{amsthm}								%Proof Environment
\usepackage{dsfont}									%Number Systems
\usepackage[dvipsnames]{xcolor}				%Colors
\usepackage[hidelinks]{hyperref} 			%Citations
\usepackage{latexsym}								%Symbolse
\usepackage{mathrsfs}								%Script font
\usepackage{mathtools}								%shortintertext in align*, arrows
\usepackage{scrextend}								%Addmargin environment
\usepackage{soul}\setul{2.5pt}{.4pt}		% underline 2.5pt below contents
\usepackage{tikz} 										%Tikz
\usepackage{tikz-cd}									%Commutative Diagrams
\usepackage[utf8]{inputenc}						%Umlauts
\usepackage{enumitem}							%Customizes itemize enviro
\DeclareMathAlphabet{\mathpzc}{OT1}{pzc}{m}{it} 
\usepackage{url}
\usepackage{bm}
\usepackage{wasysym} 							%RHD




%-----------------------------------------------------------
% Margins
%-----------------------------------------------------------
\usepackage[left=1in,top=1in,right=1in,bottom=1in,head=.2in]{geometry}

% Display Mathmode Padding
\makeatletter
\g@addto@macro \normalsize {
	\setlength\abovedisplayskip{2pt plus 0pt minus 2pt}
	\setlength\belowdisplayskip{2pt plus 0pt minus 2pt}}
\makeatother



%-----------------------------------------------------------
% Fonts
%----------------------------------------------------------- 
%\usepackage{libertine}
%\usepackage{mathpazo}
%\usepackage{newtxtext}
%\usepackage{kpfonts}
%\linespread{1.2} 			% Changes line spacing



%------------------------------------------------------------
% Isaac's Custom Environments
%------------------------------------------------------------
\swapnumbers
\newtheorem{theorem}{Theorem}[section] % numbered theorems, lemmas, etc
\newtheorem{lemma}[theorem]{Lemma}
\newtheorem{proposition}[theorem]{Proposition}
\newtheorem{corollary}[theorem]{Corollary}
\newtheorem*{theorem*}{Theorem}
\newtheorem*{lemma*}{Lemma}
\newtheorem*{proposition*}{Proposition}
\newtheorem*{corollary*}{Corollary}

\theoremstyle{definition}
\newtheorem{definition}[theorem]{Definition}
\newtheorem{remark}[theorem]{Remark}
\newtheorem{remarks}[theorem]{Remarks}
\newtheorem{example}[theorem]{Example}
\newtheorem{examples}[theorem]{Examples}
\newtheorem*{definition*}{Definition}
\newtheorem*{remark*}{Remark}
\newtheorem*{remarks*}{Remarks}
\newtheorem*{example*}{Example}
\newtheorem*{examples*}{Examples}



%------------------------------------------------------------
% Isaac's Custom Environments
%------------------------------------------------------------
\renewenvironment{proof}{\underline{Proof}.}{\qed}



%------------------------------------------------------------
% Isaac's Custom Commands
%------------------------------------------------------------

%%%% - renewed fonts - %%%%
\renewcommand\emptyset{\varnothing}
\renewcommand\geq{\geqslant}
\renewcommand\leq{\leqslant}
\renewcommand\qedsymbol{\small $\square$}
\renewcommand\hat{\widehat}
\renewcommand\tilde{\widetilde}
\renewcommand\:{\colon}
\renewcommand\bar[1]{\overline{#1}}

%%%% - math cal - %%%%
\newcommand{\calA}{\mathcal{A}}
\newcommand{\calB}{\mathcal{B}}
\newcommand{\calC}{\mathcal{C}}
\newcommand{\calD}{\mathcal{D}}
\newcommand{\calE}{\mathcal{E}}
\newcommand{\calF}{\mathcal{F}}
\newcommand{\calH}{\mathcal{H}}
\newcommand{\calI}{\mathcal{I}}
\newcommand{\calL}{\mathcal{L}}
\newcommand{\calM}{\mathcal{M}}
\newcommand{\calN}{\mathcal{N}}
\newcommand{\calO}{\mathcal{O}}
\newcommand{\calP}{\mathcal{P}}
\newcommand{\calR}{\mathcal{R}}
\newcommand{\calS}{\mathcal{S}}
\newcommand{\calT}{\mathcal{T}}
\newcommand{\calW}{\mathcal{W}}

%%%% - math ds - %%%%
\newcommand{\C}{\mathds{C}}
\newcommand{\E}{\mathds{E}}
\newcommand{\N}{\mathds{N}}
\newcommand{\Q}{\mathds{Q}}
\newcommand{\R}{\mathds{R}}
\newcommand{\Z}{\mathds{Z}}

%%%% - misc - %%%%
\newcommand{\green}[1]{\textcolor{green}{#1}}
\newcommand{\orange}[1]{\textcolor{orange}{#1}}
\newcommand{\purple}[1]{\textcolor{Fuchsia}{#1}}
\newcommand{\red}[1]{\textcolor{red}{#1}}

%%%% - specific - %%%%
\newcommand{\1}{\mathds{1}}
\newcommand{\Alt}{\textnormal{Alt}}
\newcommand{\Aut}{\textnormal{Aut}}
\newcommand{\coker}{\textnormal{coker}}
\newcommand{\core}{\textnormal{core}}
\renewcommand{\d}[2]{[#1 \hspace{-.2em} : \hspace{-.2em} #2]}
\newcommand{\End}{\textnormal{End}}
\newcommand{\Ext}{\textnormal{Ext}}
\newcommand{\Free}{\textnormal{Free}}
\newcommand{\Gal}[2]{\textnormal{Gal}(#1 \hspace{-.2em} : \hspace{-.2em} #2)}
\newcommand{\Grp}{\textnormal{Grp}}
\newcommand{\Hom}{\textnormal{Hom}}
\newcommand{\im}{\textnormal{im}}
\newcommand{\Homeo}{\textnormal{Homeo}}
\newcommand{\hTop}{\textnormal{hTop}}
\newcommand{\Inn}{\textnormal{Inn}}
\newcommand{\Mod}{\textnormal{Mod}}
\DeclareMathOperator*{\moplus}{\text{\raisebox{0.25ex}{\scalebox{0.8}{$\bigoplus$}}}}
\DeclareMathOperator*{\Moplus}{\text{\raisebox{0.35ex}{\scalebox{0.6}{$\bigoplus$}}}}
\DeclareMathOperator*{\motimes}{\text{\raisebox{0.25ex}{\scalebox{0.8}{$\bigotimes$}}}}
\DeclareMathOperator*{\Motimes}{\text{\raisebox{0.35ex}{\scalebox{0.6}{$\bigotimes$}}}}
\newcommand{\obj}{\textnormal{obj}}
\newcommand{\rank}{\textnormal{rank}}
\newcommand{\Rng}{\textnormal{Rng}}
\newcommand{\Set}{\textnormal{Set}}
\newcommand{\Sub}{\textnormal{Sub}}
\newcommand{\Syl}{\textnormal{Syl}}
\newcommand{\Sym}{\textnormal{Sym}}
\newcommand{\T}{\mathpzc{T}}
\newcommand{\Top}{\textnormal{Top}}
\newcommand{\Tor}{\textnormal{Tor}}
\newcommand{\Torsion}{\textnormal{Torsion}}
\newcommand{\exc}[1]{\vspace{-2.5pt}\begin{itemize}[leftmargin=15pt]\item[$\RHD$] \textit{\textbf{Exercise}. #1}\end{itemize}}

\setlist[itemize]{label=$\hspace{1pt} \tiny\textbullet[1pt]$}

\let\oldtextbullet\textbullet
\renewcommand{\textbullet}[1][0pt]{%
  \mathrel{\raisebox{#1}{$\bullet$}}%
}




% footnotes %
\renewcommand{\thefootnote}{\fnsymbol{footnote}}
\newcommand{\foot}[1]{\setcounter{footnote}{1}\footnote{\ #1}}

% Title
\title{Graduate Topology Notes}
\author{\vspace{-10pt}Isaac}

% Header
\usepackage{fancyhdr}
\usepackage{mathrsfs}
\pagestyle{fancy}
\fancyhf{}
\fancyhead[CO]{\small\textsc{Topology}}
\fancyhead[CE]{\small\textsc{I. Milan}}
\cfoot{\ \vskip.01in $_{\thepage}$}

% Contact Information
%\address{Bryn Mawr College,
%Bryn Mawr, PA 19003}
%\email{icraig@brynmawr.edu} 



\begin{document}

\begin{abstract}
	\vspace{-10pt}These are some of the notes I took during my graduate topology courses. They cover the basics for Point-set, Algebraic, and Differential Topology. They emphasizing the definitions, big theorems, and exercises for each theory. Hard/unintuitive exercises are marked with an asterisk.
\end{abstract}

\maketitle
\vspace{-15pt}

\setcounter{tocdepth}{1}
\tableofcontents



\section*{Point-Set Topology}

\subsection*{\underline{Basic Topology}}

\begin{definition*}
	In a space $X$, an \textbf{open neighborhood} of a point $x \in X$ is an open set $U$ such that $x \in U \subseteq X$; a \textbf{neighborhood} of a point $x \in X$ is a set $V \subseteq X$ containing an open neighborhood; i.e. there exists an open $U$ satisfying $x \in U \subseteq V$.
\end{definition*}

\subsection*{\underline{Quotient Spaces}}

\begin{theorem*}
	\textnormal{(Characteristic Property of Quotient Maps)} If $p\: X \to Y$ is a quotient map, a map $f\: Y \to Z$ is continuous if and only if $f \circ p$ is continuous.
\end{theorem*}

\begin{center}
	\begin{tikzcd}
		X \arrow[d, "p"'] \arrow[dr, "fp"]& \\
		Y \arrow[r, "f"'] & Z.
	\end{tikzcd}
\end{center}

\vskip20pt



\subsection*{\underline{Connectedness}}

\begin{definition*}
	A space $X$ is \textbf{connected} if there does not exist a \textbf{separation} of $X$, i.e. a pair of nonempty, disjoint, open sets $U, V \subset X$ covering $X$.
\end{definition*}

\begin{definition*}
	A space $X$ is \textbf{locally connected} if every open neighborhood of every point contains a connected open neighborhood.
\end{definition*}

\begin{definition*}
	A space $X$ is \textbf{path-connected} if for every pair of points $a, b \in X$, there exists a path $\gamma\: I \to X$ from $a$ to $b$, i.e. $\gamma(0) = a$ and $\gamma(1) = b$.
\end{definition*}

\begin{definition*}
	A space $X$ is \textbf{locally path-connected} if every open neighborhood of every point contains a path-connected open neighborhood.
\end{definition*}

\exc{Prove $I = [0,1]$ is connected. Conclude that path-connected implies connected.}
\exc{Give an example of a space which is path-connected, but not locally path-connected.}
\exc{Give an example of a space which is connected, but not path-connected.}
\exc{Prove that every connected, locally path-connected space is path-connected.}

\vskip20pt



\subsection*{\underline{Compactness}}

\begin{definition*}
	A space $X$ is \textbf{compact} if every open cover of $X$ has a finite subcover.
\end{definition*}

\begin{definition*}
	A space $X$ is \textbf{sequentially-compact} if every sequence has a convergent subsequence.
\end{definition*}

\begin{theorem*}
	\textnormal{(Heine-Borel)} A subset of $\R^n$ is compact if and only if it is closed and bounded.
\end{theorem*}

\begin{theorem*}
	\textnormal{(Bolzano-Weierstrass)} A bounded sequence in $\R^n$ has a convergent subsequence.
\end{theorem*}

\exc{Show every closed subspace of a compact space is compact.}
\exc{Show every compact subspace of a Hausdorff space is closed.}
\exc{Show $I = [0,1]$ is compact. Conclude that Heine-Borel holds.}
\exc{Prove Bolzano-Weierstrass; use this theorem to show a subset of $\R^n$ is sequentially-compact if and only if it is closed and bounded.$^*$}
\exc{Prove compact and sequential compactness are equivalent in metric spaces.*}

\begin{definition*}
	A space $X$ is \textbf{locally compact} if every point has a compact neighborhood.
\end{definition*}

\vskip20pt



\subsection*{\underline{Proper Maps}}

\begin{definition*}
	A map between spaces is \textbf{proper} if the preimage of every compact subspace is compact.
\end{definition*}

\exc{Let $Y$ be a locally compact, Hausdorff space. Prove a proper map $X \to Y$ is closed.}
\exc{Give an example of a proper map whose image is not closed.}

\vskip20pt



\subsection*{\underline{Metrizability}}

\begin{definition*}
	A space with a metric is said to be \textbf{metrizable}.
\end{definition*}

\begin{theorem*}
	\textnormal{(Urysohn's Theorem)} A regular, Hausdorff, second-countable space is metrizable.
\end{theorem*}

\exc{Prove that manifolds are metrizable.}

\vskip40pt





\section*{Algebraic Topology}

\subsection*{\underline{Homotopy Basics}}

\begin{definition*}
	A map is \textbf{nullhomotopic} if it is homotopic to a constant map.
\end{definition*}

\begin{definition*}
	A \textbf{retract} of $X$ onto $A \subseteq X$ is a map $r\: X \to X$ with $r(X) = A$ and $r|_A = \1_A$.
\end{definition*}

\begin{definition*}
	A \textbf{deformation retract} of a space $X$ onto a subspace $A$ is a homotopy $f_t\: X \to X$ relative to $A$ with $f_0 = \1_X$ and $f_1(X) = A$.
\end{definition*}

\exc{Relate deformation retract and retract in terms of homotopic maps.}

\begin{definition*}
	A map $f\: X \to Y$ is a \textbf{homotopy equivalence} if there is a map $g\: Y \to X$ such that $fg \simeq \1$ and $gf \simeq \1$. In such a case, $X$ and $Y$ are said to be \textbf{homotopy equivalent} or to have the same \textbf{homotopy type}.
\end{definition*}

\begin{definition*}
	A space having the homotopy type of a point is called \textbf{contractible}.
\end{definition*}

\exc{Prove that a space is contractible if its identity map is nullhomotopic.}
\exc{Prove if $Y$ is contractible, $[X, Y]$ is trivial.}
\exc{Prove that if $X$ is contractible and $Y$ is path-connected, then $[X, Y]$ is trivial.}

\vskip20pt



\subsection*{\underline{Fundamental Group}}

\begin{definition*}
	The \textbf{fundamental group} of $X$ at $x \in X$ is the group $\pi_1(X, x) = [(S^1,s), (X,x)]$ under concatenation. Given a map $f\: X \to Y$, the map $f_*\: \pi_1(X,x) \to \pi_1(Y,y)$  defined by $f_*([\gamma]) = [f \circ \gamma]$ is called the \textbf{induced homomorphism} of $f$ on the fundamental group.
\end{definition*}

\begin{definition*}
	A space is \textbf{simply connected} if its fundamental group is trivial.
\end{definition*}

\exc{If $X$ retracts onto $A$, prove the inclusion induced homomorphism is injective. If $X$ deformation retracts onto $A$, prove this map is an isomorphism.}
\exc{If $X \to Y$ is a homotopy equivalence, prove the induced map is an isomorphism.}

\begin{theorem*}
	\textnormal{(van Kampen)} Let $X$ be a space; $A, B \subseteq X$ path-connected; $A \cap B$ nonempty and path-connected; $x \in A \cap B$. Then $\pi_1(X, x)$ is the \textbf{free product with amalgamation} of $\pi_1(A, x)$ with $\pi_1(B, x)$, written $\pi_1(A, x) *_{\pi_1(A \cap B, x)} \pi_1(B, x)$. That is, given group presentations
	\begin{align*}
		\pi_1(A, x) &= \langle a_1, \dots, a_\ell \ | \ r_1, \dots, r_p\rangle \\
		\pi_1(B, x) &= \langle b_1, \dots, b_m \ | \ s_1, \dots, s_q \rangle \\
		\pi_1(A \cap B, x) &= \langle c_1, \dots, c_n \ | \ t_1, \dots, t_r \rangle 
	\end{align*}
	and inclusion maps $i\: A \to X$ and $j\: B \to X$, we have a representation
	\[ \pi_1(X, x) \cong \langle a_1, \dots, a_\ell, b_1, \dots, b_m \ | \ r_1, \dots, r_p, s_1, \dots, s_q, i_*(c_k) = j_*(c_k) \rangle. \]
\end{theorem*}

\exc{Show $S^n$ is simply connected for $n \geq 2$, and compute $\pi_1(P^n)$.}
\exc{Find a presentation for the fundamental group of the Klein bottle. Find a double covering map $p\: T \to K$ from the torus $T$ to $K$. Describe the induced homomorphism on fundamental groups.}
\exc{Compute $\pi_1(F_g)$, the genus $g$ closed connected orientable surface.}

\vskip20pt



\subsection*{\underline{Covering Spaces}}

\begin{definition*}
	A \textbf{covering space} of $X$ is a space $\tilde X$ with a map $p\: \tilde X \to X$ such that $X$ is covered by open sets $\{U_i\}$ having the property $p^{-1}(U_i)$ is a disjoint union of open sets in $\tilde X$, each mapped homeomorphically by $p$ onto $U_i$.
\end{definition*}

\begin{proposition*}
	\textnormal{(Homotopy Lifting in Coverings)} Let $\tilde X \to X$ be a covering space. Given a homotopy $f_t \: Y \to X$ and a lift $\tilde f_0\: Y \to \tilde X$, there exists a unique homotopy $\tilde f_t\: Y \to \tilde X$ lifting $f_t$.
\end{proposition*}

\begin{proposition*}
	The induced map of a cover space is injective; the image subgroup consists of homotopy classes of loops whose lifts are also loops.
\end{proposition*}

\begin{proposition*}
	The number of sheets of a path-connected cover space $p\: \tilde X \to X$  is the index of $p_*(\pi_1(\tilde X, \tilde x))$ in $\pi_1(X, x)$.
\end{proposition*}

\begin{proposition*}
	\textnormal{(Lifting Criterion)} Let $p\: (\tilde X, \tilde x) \to (X, x)$ be a cover space, $Y$ path-connected and locally path-connected, and $f\: (Y, y) \to (X, x)$. Then a lift of $f$ exists iff $f_*(\pi_1(Y, y)) \subset p_*(\pi_1(\tilde X, \tilde x))$.
\end{proposition*}

\exc{Calculate $\pi_1(P^n)$ again. You may assume $\pi_1(S^n) = 1$, $n > 1$.}
\exc{Prove every map $P^3 \to T^2$ is nullhomotopic. Prove $\1\: P^3 \to P^3$ is not nullhomotopic.}

\begin{definition*}
	A space is \textbf{semilocally simply-connected} if each point $x \in X$ is contained in a neighborhood $U$ whose inclusion induced map $\pi_1(U, x) \to \pi_1(X,x)$ is trivial.
\end{definition*}

\begin{theorem*}
	\textnormal{(Classification of Coverings)} Suppose $X$ is path-connected, locally path-connected, and semilocally simply-connected. Then there exists a bijection between the set of basepoint-preserving isomorphism classes of path-connected covering spaces $p\: (\tilde X, \tilde x) \to (X, x)$ and the set of subgroups $\pi_1(X, x)$, obtained by associating the subgroup $p_*(\pi_1(\tilde X, \tilde x))$ to the covering space $(\tilde X, \tilde x)$.
\end{theorem*}

\begin{definition*}
	The \textbf{universal cover} of a space $X$ is a simply-connected covering space $\tilde X$.
\end{definition*}
\exc{Show homotopic spaces have the same universal covers.$^*$}
\exc{Show $T^2$ and $S^1 \vee S^1 \vee S^2$ are not homotopy equivalent \textnormal{(}with\textnormal{(}out\textnormal{)} homotopy groups\textnormal{)}.}

\begin{definition*}
	For a covering space $p\: \tilde X \to X$, the homeomorphisms $\tilde X \to \tilde X$ are called \textbf{deck transformations}. These form a group $G(\tilde X)$ under composition.
\end{definition*}

\exc{Draw two lifts of the 2-torus, one regular and one irregular.}

\begin{definition*}
	A covering space $\tilde X \to X$ is \textbf{normal} (or \textbf{regular}) if for any pair of lifts $\tilde x_1, \tilde x_2 \in \tilde X$ of a point $x \in X$, there is a deck transformation taking $\tilde x_1$ to $\tilde x_2$.
\end{definition*}

\begin{proposition*}
	Let $p\: (\tilde X, \tilde x) \to (X, x_0)$ be a path-connected covering space of the path-connected, locally path-connected space $X$, and let $H$ be the subgroup $p_*(\pi_1(\tilde X, \tilde x)) \subset \pi_1(X, x)$. Then
	\begin{itemize}[leftmargin=*]\setlength\itemsep{0em}
		\item the covering space is normal iff $H$ is a normal subgroup;
		\item $G(\tilde X)$ is isomorphic to $N_H(\pi_1(X,x))/H$.
	\end{itemize}
	In particular, if $\tilde X$ is a normal cover, $G(\tilde X) \cong \pi_1(X,x)/H$.
\end{proposition*}

\begin{definition*}
	Given a covering space $\tilde X \to X$, the group $G(\tilde X)$ defines an action, called the \textbf{cover action}, on $\Homeo(Y)$ in the obvious way: $g \cdot Y = g(Y)$ for all $g \in G(\tilde X)$ and $Y \in \Homeo(Y)$.
\end{definition*}

\exc{Show that each $y \in Y$ has a neighborhood $U$ such that all images $g(U)$ for varying $g \in G(\tilde X)$ are disjoint. In other words, the cover action is free \textnormal{(}$g_1(U) = g_2(U)$ implies $g_1 = g_2$\textnormal{)}.$^*$}

\vskip20pt



\subsection*{\underline{Homology Basics}}

\begin{definition*}
	Given a sequence of homomorphisms of abelian groups and homomorphisms
		\[ \cdots \to C_{n+1} \xrightarrow{\partial_{n+1}} C_n \xrightarrow{\partial_n} \cdots \to C_1 \xrightarrow{\partial_1} C_0 \xrightarrow{\partial_0} 0 \]
	with $\partial_n\partial_{n+1} = 0$ for each $n$ is called a \textbf{chain complex}. Note then $\im(\partial_{n+1}) \subset \ker(\partial_n)$; we define the \textbf{$\boldsymbol{n}$-th homology group} of the chain complex to be the quotient group $H_n(C_n) = \ker(\partial_n)/\im(\partial_{n+1})$. Elements of $H_n$ are \textbf{homology classes}; elements of the kernel are \textbf{cycles}; elements of the image are \textbf{boundaries}; two cycles representing the same homology class are \textbf{homologous}.
\end{definition*}

\begin{definition*}
	Let $X$ be a simplicial complex and $\Delta_n(X)$ be the free abelian group with basis the open $n$-simplices of $X$. Elements of $\Delta_n(X)$, called \textbf{$\boldsymbol{n}$-chains}, are finite formal sums $\sum_\alpha n_\alpha \sigma_\alpha$, where $\sigma_\alpha\: \Delta^n \to X$ and $n_\alpha \in \Z$. A \textbf{boundary homomorphism} $\partial_n \: \Delta_n(X) \to \Delta_{n-1}(X)$ is defined on each simplex $\sigma_\alpha$ by 
		\[ \partial_n(\sigma_\alpha) = \sum_i (-1)^i \sigma_\alpha|_{[v_0, \dots, \hat{v_i}, \dots, v_n]}. \]
	The chain complex $(\Delta_n, \partial_n)$ induces the \textbf{$\boldsymbol{n}$-th simplicial homology group} of $X$ in the usual way.
\end{definition*}

\exc{Read and define cellular and singular homology.}
\exc{Calculate the homology for the $n$-sphere, $n$-projective space, $n$-torus, genus $g$ torus.}
\exc{Calculate the homology of the Klein bottle $K$. According to the Classification of Surfaces, all surfaces are given up to homeomorphism by
	\[ S^2, T^2, \dots, T^2 \#_m T^2, \dots, P^2, \dots, P^2 \#_m P^2, \dots. \]
Determine the surface to which $K$ is homeomorphic.}
\exc{Let $X$ be the space constructed from the torus $T^2$ and the real projective plane $P^2$ by identifying a meridinal circle on $T^2$ to a circle on $P^2$ representing the generator of $\pi_1P^2$.
\begin{itemize}[leftmargin=22pt]\setlength\itemsep{0em}
	\item[\textnormal{(a)}] Compute the homology groups of $X$;
	\item[\textnormal{(b)}] compute the fundamental group of $X$;
	\item[\textnormal{(c)}] show that $X$ has exactly two inequivalent 2-fold covers;
	\item[\textnormal{(d)}] one cover has infinite cyclic fundamental group and is homotopy equivalent to a wedge of spheres. Give a geometric description of this cover and specify the wedge of spheres to which it is homotopy equivalent.
\end{itemize}}
\exc{Calculate the homology of the space obtained by identifying the circle $S^1 \times \{1\}$ on the torus $T^2 = S^1 \times S^1$ with a meridinal circle of the Klein bottle $K$.}
\exc{Let $X$ have path components $\bigsqcup_i X_i$, prove that $H_n(X) \cong \oplus_i H_n(X_i)$.}
\exc{Show that $H_0(X) = \bigoplus \Z$, with one copy of $\Z$ for each path-component.$^*$}
\exc{\textnormal{(Dimension)} Show that $H_0(\{x\}) = \Z$ and $H_n(\{x\}) = 0$ otherwise.}

\begin{definition*}
	The \textbf{reduced homology groups} $\tilde H_n(X)$ are the homology groups of
		\[ \cdots \xrightarrow{\partial_2} C_1(X) \xrightarrow{\partial_1} C_0(X) \xrightarrow{\varepsilon} \Z \to 0, \]
	where $\varepsilon(\sum_\alpha n_\alpha \sigma_\alpha) = \sum_\alpha n_\alpha$.
\end{definition*}

\exc{Prove that $H_0(X) \cong \tilde H_0(X) \oplus \Z$ and $H_n(X) \cong \tilde H_n(X)$ otherwise.}

\begin{definition*}
	A map $f\: X \to Y$ induces a \textbf{chain homomorphism} $f_\#\: C_n(X) \to C_n(Y)$ by extending $f_\#(\sigma) = f\sigma\: \Delta^n \to Y$ linearly. If $f_\# \partial = \partial f_\#$, we say that $f_\#$ is a \textbf{chain map}.
\end{definition*}

\exc{Show that given a chain map $f_\#\: C_n(X) \to C_n(Y)$ the induced $f_*\: H_n(X) \to H_n(Y)$ given by $f_*[\sigma] = [f_\# \sigma] = [f\sigma]$ is a homomorphism.*}
\exc{\textnormal{(Homotopy)} Show that homotopic maps induce the same homomorphism.*}
\exc{\textnormal{(Functoriality)} Show that $\1_* = \1$ and $(fg)_* = f_*g_*$.} 
\exc{Show homotopy equivalences induce isomorphisms on homology.}

\begin{definition*}
	For a pair $(X, A)$, the \textbf{relative chain complex} of $X$ relative to $A$ is the chain complex $C_n(X, A) = C_n(X)/C_n(A)$ with boundary map $\partial\: C_n(X, A) \to C_{n-1}(X, A)$ the induced quotient map. The resulting homology groups $H_n(X, A)$ are called the \textbf{relative homology groups} of $X$ relative to $A$. We define \textbf{relative cycles} and \textbf{relative boundaries} similarly.
\end{definition*}

\begin{definition*}
	A short exact sequence $0 \to A \xrightarrow{f} B \xrightarrow{g} C \to 0$ \textit{splits on the left} if there is a homomorphism $\ell\: B \to A$ such that $\ell f = \1_A$; it \textit{splits on the right} if there is a homomorphism $r\: C \to B$ such that $gr = \1_C$. In either case, we say the sequence \textbf{splits}.
\end{definition*}

\begin{lemma*}
	\textnormal{(SES Lemma)} If $0 \to A \xrightarrow{f} B \xrightarrow{g} C \to 0$ is a short exact sequence: \textnormal{(a)} it splits on the left iff it splits on the the right; \textnormal{(b)} it splits if $C$ is free abelian; \textnormal{(c)} if it splits, $A \moplus C \cong B$.
\end{lemma*}
\begin{lemma*}
	\textnormal{(ES Lemma)} If $\dots \xrightarrow{p} A \xrightarrow{q} B \xrightarrow{r} C \xrightarrow{s} \dots$ is exact, there is a short exact sequence
		\[ 0 \to \coker(p) \to B \to \ker(s) \to 0 \]
	where $\coker(p) = B/\im(p)$.
\end{lemma*}

\begin{theorem*}
	\textnormal{(Long Exact Sequence)} For any pair $(X, A)$ we have a long exact sequence
		\begin{equation}\label{LES} \cdots \xrightarrow{\partial_{n+1}} \tilde H_n(A) \xrightarrow{i_*} \tilde H_n(X) \xrightarrow{j_*}\tilde H_n(X, A) \xrightarrow{\partial} H_{n-1}(A) \xrightarrow{i_*} \cdots \to \tilde H_0(X, A) \to 0 \end{equation}
	where $i\: A \hookrightarrow X$ is an inclusion map and $j\: X \to X/A$ is a quotient map.
\end{theorem*}

\exc{Show that a retract $r\: X \to A$ induces a split short exact sequence on homology. Conclude that $H_n(X) \cong H_n(A) \oplus H_n(X, A)$ (hint: use the Splitting Lemma).}
\exc{Calculate the homology groups of $S^n$ with this theorem.}

\begin{theorem*}
	\textnormal{(Excision)} Given subspaces $Z \subset A \subset X$ such that the closure of $Z$ is contained in the interior of $A$, then the inclusion $(X - Z, A - Z) \hookrightarrow (X, A)$ induces isomorphisms on homology.
\end{theorem*}

\begin{definition*}
	A pair $(X, A)$ consisting of a space $X$ with nonempty, closed subspace $A$ that is a deformation retract of some neighborhood in $X$ is called a \textbf{good pair}.
\end{definition*}

\begin{proposition*}
	For good pairs $(X, A)$, the quotient map $(X, A) \to (X/A, A/A)$ induces isomorphisms $H_n(X, A) \cong H_n(X/A, A/A) \cong \tilde H_n(X/A)$ for all $n$.
\end{proposition*}

\exc{Prove that if nonempty, open sets $U \subset \R^n$, $V \subset \R^m$ are homeomorphic, then $n = m$ \textnormal{(}hint: use excision with $Z = \R^n \setminus U$\textnormal{)}.}

\begin{theorem*}
	The long exact sequence of a pair is \textbf{natural}; given $f\: (X, A) \to (Y, B)$,
	\begin{center}
		\begin{tikzcd}
			\cdots \arrow[r] & H_n(A) \arrow[r, "i_*"]\arrow[d, "f_*"] & H_n(X) \arrow[r, "j_*"]\arrow[d, "f_*"] & H_n(X, A) \arrow[r, "\partial"]\arrow[d, "f_*"] & H_{n-1}(A) \arrow[r] \arrow[d, "f_*"] & \cdots \\
			\cdots \arrow[r] & H_n(B) \arrow[r, "i_*"] & H_n(Y) \arrow[r, "j_*"] & H_n(Y, B) \arrow[r, "\partial"] & H_{n-1}(B) \arrow[r] & \cdots
		\end{tikzcd}
	\end{center}
	is a commutative diagram.
\end{theorem*}

\exc{\textnormal{(Naturality)} Show that $(f|_A)_*\partial = \partial f_*$. Conclude this theorem.}

\vskip20pt



\subsection*{\underline{More Homology}}

\begin{definition*}
	For a finite CW complex $X$, the \textbf{Euler characteristic} of $X$ is the integral value $\chi(X) = \sum (-1)^n c_n$, where $c_n$ is the number of $n$-cells in $X$.
\end{definition*}

\exc{Show that $\chi(X) = \sum (-1)^n \ \rank(H_n(X))$.}
\exc{Recall that every closed connected orientable surface $F$ is homeomorphic to a connected-sum of copies of the torus. The number of copies is called the \textit{genus} of $F$.
\begin{itemize}[leftmargin=20pt]\setlength\itemsep{0em}
	\item[\textnormal{(a)}] Find the genus of a surface in terms of Euler characteristic. No proof is required;
	\item[\textnormal{(b)}] Let $F$ be a surface of genus $g$, and $E$ be an $n$-fold covering space of $F$. Derive a formula for the genus $h$ of $E$ in terms of $g$ and $n$.
	\item[\textnormal{(c)}] Draw pictures of two 3-fold covering spaces of a surface $F$ of genus 2, one regular and one irregular. Indicate in your pictures the curve in $F$ along which the connected-sum is performed, and its lifts in the covers.
\end{itemize}}

\begin{theorem*}
	\textnormal{(Mayer-Vietoris)} If $A, B$ are subspaces of $X$ whose interiors cover $X$, then 
		\[ \cdots \xrightarrow{\partial} H_n(A \cap B) \to H_n(A) \oplus H_n(B) \to H_n(X) \xrightarrow{\partial} H_{n-1}(A \cap B) \to \cdots \to H_0(X) \to 0 \]
	is an exact sequence.
\end{theorem*}

\exc{Calculate $H_k(S^n)$ and $H_k(T^n)$ using Mayer-Vietoris.}

\begin{definition*}
	Let $G$ be an abelian group. There is no obstruction to repeating the construction for simplicial homology of $X$ so that $n_i \in G$. The resulting homology groups $H_n(X; G)$ are called the \textbf{homology groups with coefficients in $\boldsymbol{G}$}. 
\end{definition*}

\begin{definition*}
	A \textbf{homology theory} is a pair of functors $H$ and $\partial$ assigning
	\begin{itemize}[leftmargin=15pt]\setlength\itemsep{0em}
		\item each pair of spaces $(X, A)$ a sequence $H_*(X, A)$ of abelian groups $H_n(X, A)$;
		\item each pair of spaces $(X, A)$ a sequence $\partial_*$ of homomorphisms $H_{n+1}(X, A) \xrightarrow{\partial_n} H_n(A)$;
		\item each map $f\: (X, A) \to (Y, B)$ a sequence $f_*$ of group homomorphisms $f_n\: H_n(X, A) \to H_n(Y, B)$;
	\end{itemize}
	such that the following six axioms (referenced throughout this section) hold:
	\begin{itemize}[leftmargin=15pt]\setlength\itemsep{0em}
		\item (functoriality) $\1_* = \1$ and $(fg)_* = f_*g_*$;
		\item (naturality) $\partial f_* = (f|_A)*\partial$;
		\item (homotopy) $f \simeq g$ implies $f_* = g_*$;
		\item (exactness) for any pair $(X, A)$, the sequence (\ref{LES}) is exact;
		\item (excision) If $\bar U \subset A^\textnormal{o}$, then $(X - U, A - U) \hookrightarrow (X, A)$ is a homology equivalence;
		\item (dimension) $H_n(\textnormal{pt}) = 0$ for all $n \neq 0$.
	\end{itemize}
\end{definition*}

\exc{Prove these axioms hold for simplicial, cellular, and singular homology.*}

\vskip20pt



\subsection*{\underline{Cohomology}}

\begin{definition*}
	For a homomorphism $\alpha\: A \to B$ and abelian group $G$, the \textbf{dual homomorphism} $\alpha^*\: \Hom(B, G) \to (A, G)$ is defined by $\alpha^*(\varphi) = \varphi\alpha$.
\end{definition*}

\exc{Show that $(\alpha\beta)^* = \beta^*\alpha^*$ for homomorphisms $A \xrightarrow{\beta} B \xrightarrow{\alpha} C$.}

\begin{definition*}
	Let $(C, \partial)$ be a chain complex of free abelian groups, and let $G$ be an abelian group. Consider the \textbf{dual cochain} $(C^n, \delta^n)$ consisting of cochain groups $C^n = (C_n)^* = \Hom(C_n, G)$ and coboundary map $\delta^n = (\partial_n)^*\: C^{n-1} \to C^n$. The homology groups $H^n(C^n; G)$ of the resulting chain complex 
		\[ \cdots \xleftarrow{\hspace{1.5em}} C^{n+1} \xleftarrow{\delta^{n+1}} C^n \xleftarrow{\ \delta^n \ \ } C^{n-1} \xleftarrow{\delta^{n-1}} \cdots \xleftarrow{\ \delta^1 \ \ } C^0 \xleftarrow{\hspace{1.5em}} 0. \]
	are called the \textbf{cohomology groups}.
\end{definition*}

\exc{Show that the chain complex for a space $X$ induces a cochain complex, i.e. $\delta\delta = 0$.}
\exc{Compute cohomology of $S^n$, $P^n$, and $F_g$ by constructing their cochains.}

\begin{theorem*}
	\textnormal{(Universal Coefficient Theorem for Cohomology)} If a chain complex $(C, \delta)$ of free abelian groups has homology groups $H_n(C)$, then there are split short exact sequences
		\[ 0 \to \Ext(H_{n-1}(C), G) \to H^n(C; G) \to \Hom(H_n(C), G) \to 0. \]
	for all $n$ and all $G$. In particular, when $G = \Z$ this gives
		\[ \fbox{ $H^n(C) \cong \Free(H_n(C)) \oplus \Torsion(H_{n-1}(C)).$ } \]
\end{theorem*}

\begin{theorem*}
	\textnormal{(Universal Coefficient Theorem for Homology)} If $C$ is a chain complex of free abelian groups, then there are split short exact sequences \[ 0 \to H_n(C) \otimes G \to H_n(C; G) \to \Tor(H_{n-1}(C), G) \to 0 \] for all $n$ and all $G$.
\end{theorem*}

\begin{theorem*}
	\textnormal{(Poincar\'e Duality)} If $M$ is a closed, orientable $n$-manifold, then for all $k$ \[ H^k(M) \cong H_{n-k}(M). \]
\end{theorem*}

\begin{theorem*}
	\textnormal{(Lefschetz Duality)} If $M$ is a compact orientable $n$-manifold whose boundary is decomposed as the union of $(n-1)$-manifolds $A$ and $B$ sharing a common boundary (i.e. $A \cap B = \partial A = \partial B$), then for all $k$ 
		\[ H^k(M, A) \cong H_{n-k}(M, B). \]
	In particular, for $A = \emptyset$ and $B = \partial M$, 
\end{theorem*}

\exc{Let $M$ be a compact, orientable 3-manifold with boundary, and assume $H_1(M) = 0$. Prove that the boundary of $M$ is a disjoint union of 2-spheres. Hint: find $H_1(\partial M)$.}
\exc{Let $M$ be a connected closed orientable $4$-manifold with $H_1(M)$ finite. Show that the Euler characteristic $\chi(M) \geq 2$. Compute $H_k(M)$ and $H^k(M)$ in terms of $H_1(M)$ and $\chi(M)$.}

\vskip20pt



\subsection*{\underline{Higher Homotopy Groups}}

\begin{definition*}
	The \textbf{$\boldsymbol{n}$-th homotopy group} of $X$ at $x \in X$ is the group $\pi_n(X, x) = [(I^n, \partial I^n), (X, x)]$ under concatenation.
\end{definition*}

\exc{Show that each $\pi_n(X, x)$ is abelian for $n > 1$.}
\exc{Show the induced map of a covering space is an isomorphism on $\pi_n(X,x)$ for $n > 1$.}

\begin{definition*}
	Let $I^{n-1}$ denote the face of $I^n$ with last coordinate $0$, and let $J^{n-1} = \bar{\partial I^n - I^{n-1}}$. The \textbf{$\boldsymbol{n}$-th relative homotopy groups} of $X$ relative to $A$ is the group \[ \pi_n(X, A, x) = [(I^n, \partial I^n, J^{n-1}), (X, A, x)]. \] 
\end{definition*}

\exc{Show $[f] \in \pi_n(X, A, x)$ is trivial if and only if it is homotopic \textnormal{rel} $\partial I^{n-1}$ to a map with image contained in $A$.}
\exc{Read and prove the long exact sequence of a pair $(X, A)$ in Hatcher.}

\begin{definition*}
	A space is \textbf{$\boldsymbol{n}$-connected} if its first $n$ homotopy groups vanish.
\end{definition*}

\begin{definition*}
	For any space $X$ and integer $k > 0$, there is a homomorphism $h_k\: \pi_k(X) \to H_k(X)$ called the \textbf{Hurewicz map}: given a choice of generator $u_k \in H_k(S^k)$, a homotopy class of maps $[f] \in \pi_k(X)$ is mapped to $f_*(u_k) \in H_k(X)$.
\end{definition*}

\begin{theorem*}
	\textnormal{(Hurewicz)} If $X$ is $(n-1)$-connected, the Hurewicz map $h_k\: \pi_k(X) \to H_k(X)$ is an isomorphism for all $k \leq n$, when $n \geq 2$.
\end{theorem*}

\begin{theorem*}
	\textnormal{(Whitehead)} If a map between CW-complexes induces isomorphisms on all homotopy groups, then it is a homotopy equivalence. Moreover, if this map is an inclusion, the complex deformation retracts onto its subcomplex. 
\end{theorem*}

\exc{Prove the Dunce Cap is contractible (hint: Hurewicz and Whitehead).}

\vskip40pt







\section*{Differential Topology}

\subsection*{\underline{Smooth Manifolds}}

\begin{definition*}
	A \textbf{topological $\boldsymbol{n}$-manifold} is a Hausdorff, second-countable space that is locally Euclidean (i.e. each point has a neighborhood homeomorphic to a subset of $\R^n$. 
\end{definition*}

\begin{definition*}
	A \textbf{coordinate chart} or $M$ is a pair $(U, \varphi)$ consisting of an open subset $U$ and a homeomorphism $\varphi\: U \to \hat U$ to an open subset $\varphi(U) = \hat U \subseteq \R^n$. We call $U$ a \textbf{coordinate domain}. We call $\varphi$ a \textbf{coordinate map}, and we denote its component functions as $\varphi(p) = (x^1(p), \dots, x^n(p))$
\end{definition*}

\exc{Show that graphs of continuous functions, spheres, and projective spaces are topological manifolds.}
\exc{Show that the product of manifolds is a manifold.}
\exc{Show that a manifold is metrizable, with countably many components.}

\begin{definition*}
	Two coordinate charts $(U, \varphi)$, $(V, \psi)$ are \textbf{smoothly compatible} if either $U \cap V = \emptyset$ or the transition function $\psi \circ \varphi^{-1}\: \varphi(U \cap V) \to \psi(U \cap V)$ is smooth.
\end{definition*}

\begin{definition*}
	An \textbf{atlas} for $M$ is a collection $\calA$ of charts whose coordinate domains cover $M$. A \textbf{smooth atlas} for $M$ is an atlas whose coordinate charts are smoothly compatible.
\end{definition*}

\begin{definition*}
	A smooth atlas is \textbf{maximal} if it is not properly contained in any larger smooth atlas. A \textbf{smooth structure} or $M$ is a maximal smooth atlas.
\end{definition*}

\exc{Every smooth atlas $\calA$ for $M$ is contained in a unique maximal smooth atlas, called the \textbf{smooth structure determined by $\boldsymbol{\calA}$.}}
\exc{Two smooth atlases for $M$ determine the same smooth structure if and only if their union is a smooth atlas.}
\exc{Prove the aforementioned graphs, spheres, and projective spaces are smooth manifolds.}
\exc{Prove the smooth structure $\calA = \{(\R, \1_\R)\}$ on $\R$ is not unique.}
\exc{Prove the space $M(m \times n, \R)$ of $m \times n$ matrices with real entries is a smooth manifold, and determine its dimension.}
\exc{Let $K$ be the subspace of $M(m \times n, \R)$ consisting of all rank 2 matrices. Show that $K$ is a manifold and determine its dimension.}

\begin{definition*}
	Let $\calA$ be a smooth structure on $M$, and let $U \subseteq M$ be open. Then $U$ is an \textbf{open submanifold} of $M$ with smooth structure defined by $\calA_U = \{(V, \varphi) \in \calA : V \subseteq U\}$.
\end{definition*}

\exc{Prove that open submanifolds are manifolds.}
\exc{Prove that the general linear group, i.e. the set of all invertible $n \times n$ matrices with real entries, is a smooth $n^2$-manifold.}
\exc{Let $m < n$. Prove that the subset $M_m(m \times n, \R)$ consisting of $m \times n$ matrices with rank $m$ is a smooth manifold.}

\vskip20pt



\subsection*{\underline{Smooth Maps}}

\begin{definition*}
	Let $M$ be a smooth $n$-manifold. A function $f\: M \to \R^k$ is a \textbf{smooth function} at \ $p \in M$ if there is a chart $(U, \varphi)$ containing $p$ whose \textbf{coordinate representative} $\hat f = f \circ \varphi^{-1}\: \hat U \to \R^k$ is smooth.
\end{definition*}

\exc{Given a smooth function, show its coordinate representation is smooth with respect to \textnormal{any} chart $(U, \varphi)$ for $M$.}

\begin{definition*}
	Let $M$ and $N$ be a smooth manifolds. A map $F\: M \to N$ is a \textbf{smooth map} if for every $p \in M$ there are charts $(U, \varphi)$ containing $p$ and $(V, \psi)$ containing $f(p)$ such that $F(U) \subseteq V$ and the \textbf{coordinate representative} $\hat F = \psi \circ F \circ \varphi^{-1}$ is smooth from $\varphi(U) \to \psi(V)$.
\end{definition*}

\exc{Prove every smooth map is continuous. Prove smoothness is a local condition (i.e. $F$ is smooth iff each point $p \in M$ has a neighborhood $U$ such that $F|_U$ is smooth).}
\exc{Prove that constant, identity, inclusion, and projection maps are smooth.}

\begin{definition*}
	A smooth map with smooth inverse is called a \textbf{diffeomorphism}.
\end{definition*}

\begin{theorem*}
	\textnormal{(Extension Lemma)} Suppose $M$ is a smooth manifold, $A \subseteq M$ is a closed subset, and $f\: A \to \R^k$ is a smooth function. For any open subset $U$ containing $A$, there exists a smooth function $\tilde f\: M \to \R^k$ supported in $U$ such that $\tilde f|_A = f$.
\end{theorem*}

\begin{theorem*}
	\textnormal{(Level Sets)} Let $M$ be a smooth manifold with closed subset $K$. There is a smooth nonnegative function $f\: M \to \R$ such that $f^{-1}(0) = K$.
\end{theorem*}
\vskip20pt



\subsection*{\underline{Tangent Vectors}}

\begin{definition*}
	A linear map $w\: C^\infty(M) \to \R$ is a \textbf{derivation} at $p \in M$ if it satisfies the product rule $w(fg) = f(p)w(g) + g(p)w(f)$. The \textbf{tangent space} of $M$ at $p$ is the set of all derivations at $p$, denoted $T_p M$. An element of $T_p M$ is called a \textbf{tangent vector} at $p$.
\end{definition*}

\exc{Show that $T_p(\R^n)$ is a vector space. Note that $T_p(\R^n) \cong \R^n$ (see Lee's text).}
\exc{Prove: if $f$ is constant, then $vf = 0$; if $f(p) = g(p) = 0$, then $v(fg) = 0$.}

\begin{definition*}
	For smooth manifolds $M$ and $N$ and smooth map $F\: M \to N$, the \textbf{differential} of $F$ at $p \in M$ is the map
		\[ dF_p\: T_pM \to T_{f(p)} N, \]
	defined by $dF_p(v)(f) = v(f \circ F)$ for each $v \in T_p(M)$ and each $f \in C^\infty(N)$.
\end{definition*}

\exc{Prove the differential is linear, maps the identity on $M$ to the identity on $T_pM$, distributes over composition, and works well with inverses.}
\exc{Let $U \subseteq M$ be open. Prove the inclusion induced differential $d\iota_p\: T_pU \to T_pM$ is an isomorphism for all $p \in U$.*}
\exc{Prove the dimension of the tangent space agrees with the dimension of the manifold.}

\begin{proposition*}
	Let $M$ be a smooth $n$-manifold and $p \in M$. Then $T_pM$ is an $n$-dimensional vector space: for any smooth chart $(U, (x^i))$ containing $p$, the \textbf{coordinate vectors} $\partial/\partial x^i\vert_p$, defined by
		\[ \frac\partial{\partial x^i}\bigg\vert_p = (d\varphi_p)^{-1}\bigg( \frac\partial{\partial x^i}\bigg\vert_{\varphi(p)} \bigg) = (d\varphi^{-1})\bigg( \frac\partial{\partial x^i}\bigg\vert_{\varphi(p)} \bigg), \]
	form a basis for $T_pM$, which we call a \textbf{coordinate basis}.
\end{proposition*}

\begin{proposition*}
	The differential $dF_p$ is represented in coordinate bases by the Jacobian matrix of the coordinate representative of $F$.
\end{proposition*}

\begin{definition*}
	The \textbf{tangent bundle} of $M$, denoted $TM$, is the space $TM = \bigsqcup_p T_pM$. A natural \textbf{projection map} $\pi\: TM \to M$ defined by $(p, v) \mapsto p$ defines a fiber bundle.
\end{definition*}

\begin{theorem*}
	The tangent bundle $TM$ has a natural topology and smooth structure, making it a $2(\dim M)$-dimensional smooth manifold. With respect to this smooth structure, the projection map is smooth.
\end{theorem*}

The smooth structure on $TM$ is defined as follows: given a chart $(U, \varphi)$ for $M$, define a chart $(\pi^{-1}(U), \tilde \varphi)$ for $TM$ with $\tilde \varphi$ defined by
	\[ \tilde \varphi \bigg( v^i \frac\partial{\partial x^i}\bigg\vert_p \bigg) = (x^1(p), \dots, x^n(p), v^1, \dots, v^n). \]
Note that with respect to charts $(U, \varphi)$ for $M$ and $(\pi^{-1}(U), \tilde \varphi)$ for $TM$, the coordinate representative of the projection is $\pi(x, v) = x$.

\begin{definition*}
	The \textbf{global differential} of a smooth map $F\: M \to N$ is the map $dF\: TM \to TN$ whose restriction to each tangent space $T_pM$ is the differential $dF_p$ at $p \in M$.
\end{definition*}

\exc{Prove the global differential is a smooth map.$^*$}
\exc{Prove that the global differential induces a covariant functor; that is, the \textbf{tangent functor} is a covariant functor from the category of smooth manifolds to the category of smooth vector bundles: to each smooth manifold $M$ it assigns the tangent bundle $TM \to M$, and to each smooth map $F\: M \to N$ it assigns the pushforward $F^*\: TM \to TN$.$^*$}

\vskip20pt




\subsection*{\underline{Immersions and Submersions}}

\begin{definition*}
	The \textbf{rank} of a smooth map $F\: M \to N$ at $p \in M$ is defined as the rank of the linear map $dF_p$. If $F$ has the same rank at each point in $M$, we say it has \textbf{constant rank}.
\end{definition*}

\begin{definition*}
	A map $f\: M \to N$ between smooth manifolds is an \textbf{immersion}/\textbf{submersion} if at each $p \in M$, the differential $dF_p$ is injective/surjective, respectively.
\end{definition*}

\exc{Prove that projections are submersions. Prove a smooth curve $\gamma\: I \to M$ is an immersion if and only if $\gamma'(t) \neq 0$ for all $t \in I$.}

\begin{theorem*}
	\textnormal{(Inverse Function Theorem)} Suppose $F\: M \to N$ is a smooth map with $dF_p$ invertible. There are connected neighborhoods $U$ and $V$ of $p$ and $F(p)$ with $F|_U$ a diffeomorphism onto $V$.
\end{theorem*}

\exc{Prove this theorem; you may use the Inverse Function Theorem for $\R^n$.}

\begin{theorem*}
	\textnormal{(Rank Theorem)} Suppose $F\: M^m \to N^n$ is a smooth map of constant rank. For each $p \in M$, there are smooth charts $(U, \varphi)$ centered at $p$ and $(V, \psi)$ centered at $F(p)$ such that $F(U) \subseteq V$, in which $F$ has coordinate representation
		\[ \hat{F}(x^1, \dots, x^r, x^{r+1}, \dots, x^m) = (x^1, \dots, x^r, 0, \dots, 0). \]
	In particular, if $F$ is a smooth submersion, this becomes
		\[ \hat{F}(x^1, \dots, x^n, x^{n+1}, \dots, x^m) = (x^1, \dots, x^n), \]
	and if $F$ is a smooth immersion, it is
		\[ \hat{F}(x^1, \dots, x^m) = (x^1, \dots, x^m, 0, \dots, 0). \]
\end{theorem*}

\exc{Prove the Rank Theorem \textnormal{(}hint: for immersions and submersions construct functions on $\R^m$ and $\R^n$, respectively, with which you can apply the Inverse Function Theorem\textnormal{)}.}
\exc{Prove if $X$ is compact and $Y$ is connected, every submersion $X \to Y$ is surjective.}
\exc{Show that there exist no submersions of compact manifolds into Euclidean spaces.}

\begin{theorem*}
	\textnormal{(Global Rank Theorem)} If $F\: M \to N$ is a surjective/injective smooth map of constant rank, then it is a smooth submersion/immersion, respectively.
\end{theorem*}

\exc{Prove the Global Rank Theorem for immersions \textnormal{(}hint: use Rank Theorem\textnormal{)}.}

\vskip20pt



\subsection*{\underline{Embeddings}}

\begin{definition*}
	A \textbf{smooth embedding} is a smooth immersion $F\: M \to N$ that is also a topological embedding (i.e. a homeomorphism onto its image).
\end{definition*}

\exc{Show that inclusions of open submanifolds $U \hookrightarrow M$ are smooth embeddings; show inclusions into products $M_j \hookrightarrow M_1 \times \dots \times M_k$ are smooth embeddings ($1 \leq j \leq k$).}
\exc{Give an example of a topological embedding that is not a smooth immersion; give an example of a smooth immersion that is not a topological embedding.}

\begin{theorem*}
	\textnormal{(Local Embedding Theorem)} A smooth map $F\: M \to N$ is a smooth immersion if and only if every point in $M$ has a neighborhood $U \subseteq M$ such that $F|_U$ is a smooth embedding.
\end{theorem*}

\begin{definition*}
	A \textbf{section} of a continuous map $\pi\: M \to N$ is a continuous right inverse. A \textbf{local section} is a section on some open subset $U \subseteq N$. 
\end{definition*}

\begin{theorem*}
	\textnormal{(Local Section Theorem)} A smooth map $\pi\: M \to N$ is a smooth submersion if and only if every point of $M$ is in the image of a smooth local section of $\pi$.
\end{theorem*}

\exc{Show that smooth submersions are open maps, and conclude that surjective submersions are quotient maps (do this with and without the Local Section Theorem).}

\begin{theorem*}
	\textnormal{(Characteristic Property of Surjective Smooth Submersions)} If $\pi\: M \to N$ is a surjective smooth submersion, a map $F\: N \to P$ is smooth if and only if $F \circ \pi$ is smooth.
\end{theorem*}

\begin{center}
	\begin{tikzcd}
		M \arrow[d, "\pi"'] \arrow[dr, "F\pi"]& \\
		N \arrow[r, "F"'] & P.
	\end{tikzcd}
\end{center}



\subsection*{\underline{Submanifolds}}

\begin{definition*}
	An \textbf{embedded submanifold} is a subset $S \subseteq M$ that is a manifold in the subspace topology, endowed with a smooth structure with respect to which the inclusion map $S \hookrightarrow M$ is a smooth embedding. The difference of dimension $\dim M - \dim S$ is the \textbf{codimension} of $S$ in $M$.
\end{definition*}

\exc{Show that open submanifolds are embedded submanifolds with codimension 0.}
\exc{Show that the image of an embedding is a submanifold.}

\begin{definition*}
	An embedded submanifold $S \subseteq M$ is \textbf{properly embedded} if the inclusion map $S \hookrightarrow M$ is a proper map.
\end{definition*}

\exc{If $S \subseteq M$ is an embedded submanifold, prove $S$ is properly embedded if and only if it is a closed subset of $M$. Conclude every compact embedded submanifold is properly embedded.}

\begin{definition*}
	Given a map $\Phi\: M \to N$ and a point $c \in N$, we call the set $\Phi^{-1}(c)$ a \textbf{level set} of $\Phi$.
\end{definition*}

\begin{theorem*}
	\textnormal{(Constant-Rank Level Set Theorem)} If $\Phi\: M \to N$ is smooth with constant rank $r$, each level set of $\Phi$ is a properly embedded submanifold of codimension $r$ in $M$.
\end{theorem*}

\begin{definition*}
	Let $\Phi\: M \to N$ be a smooth map. A point $p \in M$ is a \textbf{regular point} of $\Phi$ if the map $d\Phi_p\: T_pM \to T_{\Phi(p)}N$ is surjective; it is a \textbf{critical point} otherwise.
\end{definition*}

\begin{definition*}
	Let $\Phi\: M \to N$ be a smooth map. A point $c \in N$ is a \textbf{regular value} of $\Phi$ if every point of the level set $\Phi^{-1}(c)$ is a regular point; it is a \textbf{critical value} otherwise. The preimage of a regular value is called a \textbf{regular level set}.
\end{definition*}

\begin{theorem*}
	\textnormal{(Regular Level Set Theorem)} Every regular level set of a smooth map is a properly embedded submanifold with codimension equal to the dimension of the codomain.
\end{theorem*}

\exc{Show that the subset of rank 1 matrices in $M(2, \R)$ is a smooth manifold.}
\exc{Show that the sphere is a smooth manifold.}

\begin{definition*}
	If $S \subseteq M$ is an embedded submanifold realized as a regular level set of a smooth map $\Phi\: M \to N$, we call $\Phi$ a \textbf{defining map} for $S$. If $U \subseteq M$ is open and $\Phi\: U \to N$ is a smooth map such that $S \cap U$ is a regular level set of $\Phi$, then $\Phi$ is called a \textbf{local defining map} for $S$.
\end{definition*}

\begin{definition*}
	Let $\iota\: S \to M$ be the inclusion map of a submanifold. The tangent space $T_pS$ is defined as the subspace $d \iota_p(T_pS) \subseteq T_pM$.
\end{definition*}

\exc{Explore the equivalent definitions of $T_pS$ using curves/derivations restricted to $S$.}

\begin{theorem*}
	If $\Phi\: U \to N$ is a local defining map for a submanifold $S \subseteq M$, then $T_pS = \ker(d\Phi_p)$ for each $p \in S \cap U$. If $S$ is a level set of $\Phi\: M \to N$, the same equality holds for each $p \in S$.
\end{theorem*}

\vskip20pt



\subsection*{\underline{Transversality}}

\begin{definition*}
	Submanifolds $X$ and $Y$ of a manifold $M$ are said to intersect \textbf{transversely} if their tangent spaces span the ambient tangent space, i.e. $T_x(X) + T_x(Y) = T_x(M)$ for each $x \in X \cap Y$.
\end{definition*}

\exc{Let $\calH = \{(x,y,z) : x^2 + y^2 - z^2 = 1\}$ and $\calS_a = \{(x,y,z) : x^2 + y^2 + z^2 = a\}$. Give equations for their intersection for varying values of $a \in \R_{\geq0}$. For which values of $a$ do they intersect transversely?}

\vskip20pt



\subsection*{\underline{Multilinear Algebra}}

\begin{definition*}
	Suppose $V_1, \dots, V_k, W$ are vector spaces; $F\: V_1 \times \dots \times V_k \to W$ is \textbf{multilinear} if 
		\[ F(v_1, \dots, v_j + av_j', \dots v_k) = F(v_1, \dots, v_j, \dots, v_n) + aF(v_1, \dots, v_j', \dots, v_k) \]
	for all scalars $a$ and all $1 \leq j \leq k$. We denote the space of all multilinear maps $V_1 \times \dots \times V_k \to W$ by $L(V_1, \dots, V_k; W)$.
\end{definition*}

\begin{definition*}
	Let $F \in L(V_1, \dots , V_k; \R)$ and $G \in L(W_1, \dots, W_\ell; \R)$. The \textbf{tensor product} of $F$ and $G$ is the element $F \motimes G \in L(V_1, \dots, V_k, W_1, \dots, W_\ell; \R)$ defined by
		\[ F \Motimes G(v_1, \dots, v_k, w_1, \dots, w_\ell) = F(v_1, \dots, v_k)G(w_1, \dots, w_\ell). \]
\end{definition*}

\exc{Show the tensor product operation is bilinear and associative.}

\begin{theorem*}
	\textnormal{(Multilinear Functions Basis)} Let $V_1, \dots, V_k$ be real vector spaces of dimension $n_1, \dots, n_k$. For each $1 \leq j \leq k$, let $(e^{(j)}_1, \dots, e^{(j)}_{n_j})$ be a basis for $V_j$ and let $(\varepsilon^1_{(j)}, \dots, \varepsilon^{n_j}_{(j)})$ be a basis for $V_j^*$;
		\[ \calB = \{ \varepsilon^{i_1}_{(1)} \motimes \dots \motimes \varepsilon^{i_k}_{(k)} : 1 \leq i_1 \leq n_1, \dots, 1 \leq i_k \leq n_k\} \]
	is a basis for $L(V_1, \dots, V_k; \R)$. 
\end{theorem*}
\vskip20pt 

\subsection*{Tensors}

\begin{definition*}
	For any set $S$, a \textbf{formal linear combination} of elements of $S$ is a function $f\: S \to \R$ such that $f(s) = 0$ for all but finitely many $s \in S$. The \textbf{free real vector space} on $S$, denoted $\calF(S)$, is the set of all formal linear combinations of elements of $S$.
\end{definition*}

\begin{proposition*}
	\textnormal{(Characteristic Property of Free Vector Spaces)} For any set $S$ and any vector space $W$, every map $f\: S \to W$ has a unique extension to a linear map $F\: \calF(S) \to W$.
\end{proposition*}

\begin{definition*}
	Let $V_1, \dots, V_k$ be real vector spaces. Let $\calR$ be the subspace of $\calF(V_1 \times \dots \times V_k)$ spanned by all elements of the following forms:
		\[ (v_1, \dots, av_i, \dots, v_k) = a(v_1, \dots, v_i, \dots v_k) \]
		\[ (v_1, \dots, v_i + v_i', \dots, v_k) - (v_1, \dots, v_i, \dots, v_k) - (v_1, \dots, v_i', \dots, v_j) \]
	with $v_i, v_i' \in V_i$, $i \in \{1, \dots, k\}$, and $a \in \R$. We define the \textbf{tensor product of $\boldsymbol{V_1}, \dots, \boldsymbol{V_k}$} to be the space 
		\[ V_1 \motimes \dots \motimes V_k = \calF(V_1 \times \dots \times V_k)/\calR, \]
	with natural projection $\Pi$. The equivalence class of an element $(v_1, \dots, v_k)$ is denoted 
		\[ v_1 \motimes \dots \motimes v_k = \Pi(v_1, \dots, v_k), \]
	and is referred to as the (abstract) \textbf{tensor product of $\boldsymbol{v_1}, \dots, \boldsymbol{v_k}$}.
\end{definition*}

\exc{Show that abstract tensor products behave multilinearly, meaning
	\[ v_1 \Motimes \dots \Motimes av_i \Motimes \dots \Motimes v_k = a(v_1 \Motimes \dots \Motimes v_i \Motimes \dots \Motimes v_k, \]
	\[ v_k \Motimes \dots \Motimes (v_i + v_i') \Motimes \dots \Motimes v_k = (v_1 \Motimes \dots \Motimes v_i \Motimes \dots \Motimes v_k) + (v_1 \Motimes \dots \Motimes v_i' \Motimes \dots \Motimes v_k). \]
}

\begin{proposition*}
	\textnormal{(Characteristic Property of Tensor Product Spaces)} Let $V_1, \dots, V_k$ be finite-dimensional real vector spaces. If $f\: V_1 \times \dots \times V_k \to X$ is any multilinear map into a vector space $X$, then there is a unique linear map $F\: V_1 \motimes \dots \motimes V_k \to X$ such that the following diagram commutes:
	\begin{center}
		\begin{tikzcd}[row sep=40pt]
			V_1 \times \dots \times V_k \arrow[r, "f"]\arrow[d, "\Pi"'] & X \\
			V_1 \motimes \dots \motimes V_k \arrow[ur, dashed, "F"']& 
		\end{tikzcd}
	\end{center}
\end{proposition*}

\begin{proposition*}
	\textnormal{(Basis of Tensor Product Space)} Let $V_1, \dots, V_k$ be real vector spaces of dimensions $n_1, \dots, n_k$. For each $j =1, \dots, k$ suppose $(E_1^{(j)}, \dots, E_{n_j}^{(j)})$ is a basis for $V_j$. Then the set
		\[ \calC = \{ E_{i_1}^{(1)} \Motimes \dots \Motimes E_{i_k}^{(k)} : 1 \leq i_1 \leq n_1, \dots, 1 \leq i_k \leq n_k \} \]
	is a basis for $V_1 \motimes \dots \motimes V_k$.
\end{proposition*}

\begin{proposition*}
	\textnormal{(Associativity of Tensor Product Space)} Let $V_1, V_2, V_3$ be finite-dimensional real vector spaces. There are unique isomorphisms
		\[ V_1 \Motimes (V_2 \Motimes V_3) \cong V_1 \Motimes V_2 \Motimes V_3 \cong (V_1 \Motimes V_2) \Motimes V_3, \]
	under which elements of the forms $v_1 \motimes (v_2 \motimes v_3)$, $v_1 \motimes v_2 \motimes v_3$, and $(v_1 \motimes v_2) \motimes v_3$ all correspond.
\end{proposition*}

\begin{proposition*}
	If $V_1, \dots, V_k$ are finite-dimensional vector spaces, there is a canonical isomorphism 
		\[ V_1^* \Motimes \dots \Motimes V_k^* \cong L(V_1, \dots, V_k; \R), \]
	under which the abstract tensor product corresponds to the tensor product of covectors.
\end{proposition*}

\vskip20pt



\subsection*{\underline{Vector Bundles}}

\begin{definition*}
	Let $\pi\: E \to M$ be a vector bundle. A \textbf{section} of $E$ is a section of the map $\pi$, that is, a continuous right inverse $\sigma\: M\to E$.
\end{definition*}

\begin{definition*}
	A \textbf{local section} of $E$ is a continuous right inverse $\sigma\: U \to E$ on some open $U \subseteq M$.
\end{definition*}

\begin{definition*}
	Let $\pi\: E \to M$ be a vector bundle. If $U \subseteq M$ is open, a $k$-tuple of local sections $(\sigma_1, \dots, \sigma_k)$ on $E$ over $U$ is \textbf{linearly independent} if their values $(\sigma_1(p), \dots, \sigma_k(p))$ form a linearly independent $k$-tuple in $E_p$ for each $p \in U$. Similarly, the are said to \textbf{span} $E$ if their values span $E_p$ for each $p \in U$.
\end{definition*}

\begin{definition*}
	A \textbf{local frame} for $E$ over $U$ is an ordered $k$-tuple $(\sigma_1, \dots, \sigma_k)$ of linearly independent local sections over $U$ that span $E$, that is, $(\sigma_1(p), \dots, \sigma_k(p))$ is a basis for $E_p$ for each $p \in U$. This is a \textbf{frame} if $U = M$.
\end{definition*}

\vskip20pt



\subsection*{\underline{Orientation}}

\begin{definition*}
	Let $V$ be a real vector space of dimension $n \geq 1$. We say ordered basis $(e_1, \dots, e_n)$ and $(\tilde e_1, \dots, \tilde e_n)$ are \textbf{consistently oriented} if the transition matrix $e_i = B_i^j \tilde e_j$ has positive determinant.
\end{definition*}

\begin{definition*}
	Let $V$ be a real vector space of dimension $n \geq 1$. An \textbf{orientation} for $V$ is an equivalence class of ordered basis, where two consistently oriented bases are equivalent. Any element in the orientation is said to be \textbf{positively oriented}; otherwise it is \textbf{negatively oriented}.
\end{definition*}

\begin{proposition*}
	Let $V$ be a real vector space of dimension $n \geq 1$. Each nonzero element $\omega \in \Lambda^n(V^*)$ determines an orientation $\calO_\omega$, the set of ordered bases $(e_1, \dots, e_n)$ such that $\omega(e_1, \dots, e_n) > 0$. We say that $\omega$ is a \textbf{positively oriented covector}.
\end{proposition*}

\begin{definition*}
	If $(E_i)$ is a local frame for $TM$, we say that $(E_i)$ is \textbf{positively oriented} if $(E_1|_p, \dots, E_n|_p)$ is a positively oriented basis for $T_pM$ at each $p \in U$.
\end{definition*}

\begin{definition*}
	A pointwise orientation is \textbf{continuous} if every point of $M$ is in the domain of an oriented local frame. An \textbf{orientation} for $M$ is a continuous pointwise orientation.
\end{definition*}

\begin{proposition*}
	Let $M$ be a smooth $n$-manifold with or without boundary. Any nonvanishing $n$-form $\omega$ on $M$ determines a unique orientation of $M$ for which $\omega$ is positively oriented at each point. Conversely, if $M$ is given an orientation, then there is a smooth nonvanishing $n$-form on $M$ that is positively oriented at each point.
\end{proposition*}

\vskip20pt



\subsection*{Tensors on Vector Spaces}

\begin{definition*}
	A \textbf{$\boldsymbol{p}$-tensor} on a real vector space $V$ is an element of the $k$-fold product $V^* \motimes \dots \motimes V^*$; that is, a real-valued multilinear function $T\: V \times \dots \times V \to \R$. A $0$-tensor is, by convention, a real number. The space of $k$-tensors is denoted $T^k(V^*)$.
\end{definition*}

\begin{theorem*}
	If $V^*$ has basis $\{\phi_1, \dots, \phi_n\}$, then 
		\[ \{\phi_{i_1} \Motimes \dots \Motimes \phi_{i_k} : 1 \leq i_1 \lneq \dots \lneq i_k \leq n\} \]
	forms a basis for $T^k(V^*)$. Consequently, $\dim T^k(V^*) = n^k$.
\end{theorem*}

\begin{definition*}
	If $T$ is a $k$-tensor and $S$ is a $\ell$-tensor, we define a $k + \ell$ tensor $T \motimes S$ by
		\[ T \Motimes S(v_1, \dots, v_k, v_{k + 1}, \dots, v_{k + \ell}) = T(v_1, \dots, v_k) \cdot S(v_{k + 1}, \dots, v_{k + \ell}). \]
	The resulting tensor $T \motimes S$ is called the \textbf{tensor product} of $T$ and $S$.
\end{definition*}

\begin{definition*}
	A $k$-tensor $T$ is \textbf{alternating} if its sign is reversed whenever variables are transposed:
		\[ T(v_1, \dots, v_i, \dots, v_j, \dots, v_k) = -T(v_1, \dots, v_j, \dots, v_i, \dots, v_k). \]
	The alternating $k$-tensors form a vector subspace $\Lambda^k(V^*)$ of $\mathpzc{T}^k(V^*)$.
\end{definition*}

\exc{Give two examples of $k$-tensors on $\R^k$ (one alternating, one not).}
\exc{Show that the tensor product is associative$^*$ and that it distributes over addition.}

\begin{definition*}
	Let $(-1)^\pi = \pm 1$ be the parity of $\pi \in S_k$. For a $k$-tensor $T$, define
		\[ T^\pi(v_1, \dots, v_k) = T(v_{\pi(1)}, \dots, v_{\pi(k)}), \hskip20pt \text{ and } \hskip20pt \Alt(T) = \frac1{k!} \sum_{\pi \in S_k} (-1)^\pi T^\pi, \]
	both of which are $p$-tensors.
\end{definition*}

\exc{Prove that $T$ is alternating if $T^\pi = (-1)^\pi T$ for all $\pi \in S_k$.}
\exc{Prove that $(T^\pi)^\sigma = T^{\pi\sigma}$ for all $\pi, \sigma \in S_k$.}
\exc{Prove $\Alt(T)$ is alternating. Prove that $T$ is alternating if and only if $\Alt(T) = T$.}

\begin{definition*}
	If $T \in \Lambda^k$, $S \in \Lambda^\ell$, we define their \textbf{wedge product} $T \wedge S = \Alt(T \motimes S) \in \Lambda^{k + \ell}$.
\end{definition*}

\begin{theorem*}
	If $V^*$ has basis $\{\phi_1, \dots, \phi_n\}$, then 
		\[ \{\phi_I = \phi_{i_1} \wedge \dots \wedge \phi_{i_k} : 1 \leq i_1 \lneq \dots \lneq i_k \leq n\} \]
	forms a basis for $\Lambda^k(V^*)$. Consequently, $\dim\Lambda^k(V^*) = \frac{n!}{k!(n - k)!}$.
\end{theorem*}


\exc{Prove the wedge is associative$^*$ and distributes over addition and scalar multiplication.}
\exc{Prove that $\phi_I = \pm\phi_J$ when two index sequences $I, J$ differ only in their orderings. Prove that $\phi_I = 0$ when any index of $I$ is repeated.}
\exc{Prove $\phi_I \wedge \phi_J = (-1)^{k\ell} \phi_J \wedge \phi_I$ for any properly defined sequences $I, J$ of length $k, \ell$ respectively. Conclude that $T \wedge S = (-1)^{k\ell} S \wedge T$ for any $T \in \Lambda^k$ and $S \in \Lambda^\ell$.}
\exc{By the previous theorem, $\dim \Lambda^k((\R^k)^*)$ is one dimensional when $k = \dim (\R^k)^*$. Why does this make sense? Consider the known alternating $k$-tensors on $(\R^k)^*$.}

\begin{definition*}
	We have, by convention, that $\Lambda^0(V^*) = \R$, and we extend the wedge product of any element in $\R$ with any tensor in $\Lambda^k(V^*)$ as the usual scalar multiplication. The wedge product then makes the direct sum
		\[ \Lambda(V^*) = \Lambda^0(V^*) \Moplus \Lambda^1(V^*) \Moplus \cdots \Moplus \Lambda^n(V^*) \]
	a noncommutative algebra, called the \textbf{exterior algebra of $\boldsymbol{V^*}$}, with identity element $1 \in \Lambda^0(V^*)$.
\end{definition*}
\vskip20pt


\begin{proposition*}
	If $v^1, \dots, v^k$ are covectors on $V$, then $v^1 \wedge \dots \wedge v^k(v_1, \dots, v_k) = \det( v^j(v_i) )$.
\end{proposition*}

\vskip20pt



\subsection*{\underline{Differential Forms}}

\begin{definition*}
	Let $X$ be a smooth manifold, and recall that an element of the vector bundle 
		\[ \Lambda^k(T^*X) = \coprod_{x \in X} \Lambda^k(T_x^*X) \]
	is an alternating $k$-form $\omega$ that assigns to each point $x \in X$ an alternating $p$-tensor $\omega(x)$ on the tangent space of $X$ at $x$. A \textbf{$\boldsymbol{p}$-form} on $X$ is a smooth section of the projection $\Lambda^k(T^*X) \to X$. The space of all $k$-forms on $X$ is denoted $\Omega^k(X)$.
\end{definition*}

\exc{Characterize $0$-forms.}
\exc{If $\phi\: X \to \R$ is a smooth function, $d\phi_x\: T_x(X) \to \R$ is a linear map, so the assignment $x \mapsto d\phi_x$ defines a $1$-form $d\phi$ on $X$. In this manner, the coordinate functions $x^i$ on $\R^k$ yield 1-forms $dx^i$ on $\R^k$. At each $z \in \R^k$, what familiar objects are the $dx^i(z)$?}

\begin{proposition*}
	Every $p$-form on an open set $U \subseteq \R^k$ may be uniquely expressed as a sum $\sum_I f_I dx^I$ over increasing index sequences $I = (i_1 < \dots, < i_p)$, where each $f_I$ is a function on $U$.
\end{proposition*}

\begin{definition*}
	If $f\: X \to Y$ is smooth and $\omega$ is a smooth $p$-form on $Y$, we define the \textbf{pullback} of $\omega$ by $f$ to be the $p$-tensor 
		\[ f^*(\omega)(x) = (df_x)^*\omega[f(x)], \]
	where $(df_x)^*$ is the dual (or transpose) map defined by $(df_x)^*\omega[f(x)] = \omega[f(x)] \circ df_x$.
\end{definition*}

\exc{Prove that $f^*(\omega_1 + \omega_2) = f^*\omega_1 + f^*\omega_2$ and prove $f^*(\omega \wedge \theta) = f^*\omega \wedge f^*\theta$ and prove $(fh)^*\omega = h^*f^*\omega$ for $p$-forms $\omega, \omega_1, \omega_2$ and a $q$-form $\theta$.}
\exc{Prove $f^*dx^i = df^i$ and conclude $f^*(\omega) = \sum_I (f^*a_I) df^I$ for $\omega = \sum_I a_I dx^I$.}
\exc{Prove that if $f\: X \to Y$ smooth and $\phi$ a chart on $Y$, then $f^*(d\phi) = d(f^* \phi)$.}

\exc{Note: one of the reasons we want to look at an $n$-form $\omega$ on an $n$-manifold $M$ is to be able to assign volume to a small region. The previous section showed these \textbf{volume forms} are essentially determinants on the tangent space. This makes sense because determinants (i) record volume of parallelepipeds and (ii) act like alternating $n$-tensors.}

\vskip20pt



\subsection*{\underline{Exterior Derivatives}}

\begin{definition*}
	If $\omega = \sum_J \omega_J dx^J$ is a smooth $k$-form on an open subset $U \subseteq \R^n$, we define its \textbf{exterior derivative} $d\omega$ to be the following $(k + 1)$-form:
		\[ d\bigg( \sum_J' \omega_Jdk^J \bigg) = \sum_J' d\omega_J \wedge dx^J, \]
	where $d\omega_J$ is just the differential of the function $\omega_J$.
\end{definition*}

\vskip20pt



\subsection*{\underline{Integration on Manifolds}}

\begin{definition*}
	Let $D \subseteq \R^n$ be a \textbf{domain of integration} (i.e. a bounded subset whose boundary has measure 0), and let $\omega = f dx^1 \wedge \dots \wedge dx^n$ be an $n$-form on $\bar D$, where $f\: \bar D \to \R$ is a continuous function. We define the \textbf{integral of $\boldsymbol{\omega}$} over $D$ to be
		\[ \int_D f dx^1 \wedge \dots \wedge dx^n =  \int_D \omega = \int_D f dx^1 \dots dx^n \]
\end{definition*}

\begin{definition*}
	Suppose $M$ is an oriented smooth $n$-manifold and $\omega$ is an $n$-form, compactly supported in the domain of a chart $(U, \varphi)$ that is either positively or negatively oriented. We define the \textbf{integral of $\boldsymbol{\omega}$ over $\boldsymbol{M}$} to be the value
		\[ \int_M \omega = \pm \int_{\varphi(U)} (\varphi^{-1})^* \omega, \]
	where the sign agrees with the positive/negative orientation.
\end{definition*}

\begin{definition*}
	Suppose $M$ is an oriented smooth $n$-manifold and $\omega$ is a compactly supported $n$-form on $M$. Let $\{U_i\}$ be a finite open cover of supp($\omega$) by domains of positively or negatively oriented charts, and let $\{\psi_i\}$ be a subordinate smooth partition of unity. Define the \textbf{integral of $\boldsymbol{\omega}$ over $\boldsymbol{M}$} to be the value
		\[ \int_M \omega = \sum_i \int_M \psi_i \ \omega \]
\end{definition*}

\exc{Show the above definition does not depend on choice of charts or partition of unity.$^*$}

\begin{theorem*}
	\textnormal{(Stokes's Theorem)} Let $M$ be an oriented smooth $n$-manifold with boundary, and let $\omega$ be a compactly supported $(n-1)$-form on $M$. Then
		\[ \int_M d\omega = \int_{\partial M} \omega. \]
\end{theorem*}





























































\end{document}



